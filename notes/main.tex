%%% Document Formatting
\documentclass[12pt,fleqn]{article}
\usepackage[a4paper,
            bindingoffset=0.2in,
            left=0.75in,
            right=0.75in,
            top=0.8in,
            bottom=0.8in,
            footskip=.25in]{geometry}
\setlength\parindent{10pt} % No indent

%%% Imports
% Mathematics
\usepackage{amsmath} % Math formatting
\numberwithin{equation}{section} % Number equation per section
\DeclareMathOperator{\Tr}{Tr}
\DeclareMathOperator{\KL}{KL}



\newtheorem{theorem}{Theorem}
\newtheorem{definition}{Definition}
\newtheorem{proof}{Proof}
\newtheorem{lemma}{Lemma}
\newtheorem{example}{Example}
\newtheorem{fact}{Fact}


\usepackage{amsmath}
\usepackage{amsfonts} % Math fonts
\usepackage{amssymb} % Math symbols
\usepackage{mathtools} % Math etc.
\usepackage{slashed} % Dirac slash notation
\usepackage{cancel} % Cancels to zero
\usepackage{empheq}
\usepackage{breqn}

\newcommand{\expect}[1]{\mathbb{E}\left[#1\right]}
\newcommand{\qed}{\hfill$\square$}


% Visualization
\usepackage{graphicx} % for including images
\graphicspath{ {} } % Path to graphics folder
\usepackage{tikz}



%%% Formating
\usepackage{hyperref} % Hyperlinks
\hypersetup{
    colorlinks=true,
    linkcolor=blue,
    filecolor=magenta,      
    urlcolor=cyan,
    pdftitle={Overleaf Example},
    pdfpagemode=FullScreen,
    }
\urlstyle{same}

\usepackage{mdframed} % Framed Enviroments
\newmdenv[ 
  topline=false,
  bottomline=false,
  skipabove=\topsep,
  skipbelow=\topsep
]{sidework} %% Side-work

\newcounter{problem}
\NewDocumentEnvironment{problem}{o}{%
  \refstepcounter{problem}%
  \IfNoValueTF{#1}
    {\def\problem@title{Problem~\theproblem}}
    {\def\problem@title{Problem~\theproblem ~(#1)}}%
  \begin{mdframed}[
    linecolor=black,
    linewidth=0.5pt,
    backgroundcolor=blue!2.5,
    innertopmargin=0pt,
    innerbottommargin=10pt,
    innerleftmargin=10pt,
    innerrightmargin=10pt,
    frametitlefont=\bfseries,
    frametitle=\problem@title,
  ]
}{%
  \end{mdframed}
} % Practice problem environment

\newcounter{answer}
\NewDocumentEnvironment{answer}{o}{%
  \refstepcounter{answer}%
  \IfNoValueTF{#1}
    {\def\answer@title{Answer~\theanswer}}
    {\def\answer@title{Answer~\theanswer ~(#1)}}%
  \begin{mdframed}[
    linecolor=black,
    linewidth=0.5pt,
    backgroundcolor=green!2.5,
    innertopmargin=0pt,
    innerbottommargin=10pt,
    innerleftmargin=10pt,
    innerrightmargin=10pt,
    frametitlefont=\bfseries,
    frametitle=\answer@title,
  ]
}{%
  \end{mdframed}
}

%%%%% ------------------ %%%%%
%%% Title
\title{Machine Learning for Physicists: Recitation Notes}
\author{Clark Miyamoto (cm6627@nyu.edu)}
\date{\today}
\begin{document}


\maketitle

\begin{abstract}
	Machine learning is important for obvious reasons, but physicist often only know the heuristics reasons for certain ML design decisions.but the exact implementation
\end{abstract}

\tableofcontents


\newpage

\part{Mathematics}
\section{Review of Linear Algebra}
Here's some linear algebra that you might not have learned in a regular class.

\subsection{Singular Value Decomposition}
Recall the eigen-decomposition of a matrix. Given a symmetric square matrix $A \in \mathbb R^{d\times d}$ with eigenvalues $\{\lambda_i\}_i$ and eigenvectors $\{e_i\}_i$. The matrix could be re-expressed as
\begin{align}
	A = U \Lambda U^T
\end{align}
where $\Lambda = \text{diag}(\lambda_1, ..., \lambda_d) \in \mathbb R^{d\times d}$ and $U \in \mathbb R^{d\times d}$ is a matrix whose columns are $\{e_i\}_i$.  

This decomposition had a lot of nice properties. In particular, $\Lambda$ is diagonal and $U$ is orthogonal. This allowed us to do all sorts of stuff easily; for example, matrix power.
\\
\\
What happens if we want to do this on non-symmetric, or even non-square matrices? Well we can use the singular value decomposition (SVD).

\subsubsection{Definitions}

\begin{definition}
	[Singular Values] Let $A \in \mathbb R^{m \times n}$. Now consider $A^T A \in \mathbb R^{n \times n}$. This is a symmetric matrix so it has positive eigenvalues $0 \leq \lambda_1 \leq ... \leq \lambda_n$. The singular values $\sigma_i$ for matrix $A$ are defined as 
	\begin{align}
		\sigma_i \equiv \sqrt{\lambda_i} , \text{ s.t. } 0 \leq \lambda_1 \leq ... \leq \lambda_n
	\end{align}
\end{definition}

\begin{fact}
	The number of non-zero singular values of $A$ correspond to the rank of $A$.
\end{fact}

\emph{Proof:} Let $A : \mathbb R^d \to \mathbb R^d$ be a linear map. Recall by Rank-Nullity theorem $\text{rank}(A) + \text{dim}\, \text{Ker}(A) = \text{dim}(\mathbb R^d)$. Recall $\text{Ker}(A) = \{v : A(v) = 0\}$, so the dimension of the kernel is the number of zero eigenvalues.

Also notice that $\text{Ker}(A) = \text{Ker}(A^T A)$. $(\implies)$ Let $v \in \text{Ker}(A)$, then $A^T A v = 0$, therefore $v \in \text{Ker}(A^T A)$. $(\impliedby)$ Let $v \in \text{Ker}(A^T A)$, then $A^T A v = 0$, meaning $x^T A^T A v = \|A v\|^2 = 0$, the vector norm is only zero when the vector is zero, therefore $Av = 0$, implying $v \in \text{Ker}(A)$. 

This means the $\text{rank}(A) = \text{dim}(\mathbb R^d) - \text{Ker}(A^T A)$. The dimension of the kernel of $A^T A$ is the number of zero singular values of $A$. 

\qed

\begin{definition}
	[SVD] $A \in \mathbb R^{m \times n}$ with singular values $0 \leq \sigma_1 \leq ... \leq \sigma_n$. Let $r$ denote the rank, or equivalently the number of singular values of $A$. The SVD of $A$ is a decomposition
	\begin{align}
		A = U \Sigma V^T
	\end{align}
	where 
	\begin{itemize}
		\item $U \in \mathbb R^{m \times m}$ orthogonal matrix
		\item $V \in \mathbb R^{n \times n}$ orthogonal matrix
		\item $\Sigma \in \mathbb R^{m \times n}$ matrix such that $[\Sigma]_{ii} = \sigma_i$ for $i \in [1, ..., r]$ and $[\Sigma]_{ii} = 0$ for $i > r$.
	\end{itemize}
\end{definition}

\begin{theorem} [Computing SVD]
	Let $A \in \mathbb R^{m \times n}$. Then $A$ has a (non-unique) SVD $A = U \Sigma V^T$,  where
	\begin{itemize}
		\item The columns of $V$ are orthonormal eigenvectors of $A^T A$, where $A^T A v_i = \sigma^2 v_i$.
		\item If $i \leq r$, s.t. $\sigma_i \neq 0$, then the $i$'th column of $U$ is given by $\sigma_i^{-1} A v_i$
	\end{itemize}
\end{theorem}

\subsubsection{Illustration of SVD}
Recall, by \href{https://www.3blue1brown.com/topics/linear-algebra}{3Blue1Brown}, matrix operations transform the coordinate space of some vector.
\begin{figure*}[h!]
	\centering
	\includegraphics[scale=0.6]{figures/linear/linear_nonilinear_transformation.png}
	\caption{Visualization from \cite{gundersen}}
\end{figure*} 
So the only hope we have at visualize the SVD is to use this intuition.\\
\\
Let $A \in \mathbb R^{2 \times 2}$.  Let $v_1, v_2 \in \mathbb R^2$ be orthonormal vectors, that is $v_i \cdot v_j = \delta_{ij}$. Say we know how $A$ acts on $v_i$, that is it rotates them to another orthonormal basis $u_i$ and rescales them according to $\sigma_i$.
\begin{align}
	A v_1 & = \sigma_1 u_1\\
	A v_2 & = \sigma_2 u_2
\end{align}
We've chosen the notations of these vectors in a very peculiar manner. You can think the SVD as just saying we map vectors in matrix $V$ to vectors in $U$ scaled by $\Sigma$.

But deriving is believing, so let's show that the matrix $A$ emits an SVD where everything lines up. 


Consider how $A$ acts on an arbitrary test vector $x$.
\begin{align}
	Mx & = M(\langle v_1, x\rangle v_1 + \langle v_2, x\rangle v_2) & \text{Basis}\\
	& =  Mv_1\, \langle v_1, x\rangle +  M v_2 \, \langle v_2, x\rangle\\
	& = \sigma_1 u_1  \langle v_1, x\rangle + \sigma_2 u_2\, \langle v_2, x\rangle  & Av_i = \sigma_i u_i\\
	& = \sigma_1 u_1 v_1^T x + \sigma_2 u_2 v_2^T x & \text{Def of inner product}\\
	& = (\sigma_1 u_1 v_1^T + \sigma_2 u_2 v_2^T) x
\end{align}
Since this holds for an arbitrary test vector
\begin{align}
	M = \sigma_1 \, u_1 v_1^T + \sigma_2 \, u_2 v_2^T = \underbrace{(u_1 \ u_2)}_{U} \underbrace{\begin{pmatrix}
		\sigma_1 & 0 \\
		0 & \sigma_2
 	\end{pmatrix}}_{\Sigma} \underbrace{\begin{pmatrix}
 		v_1^T \\ v_2^T
 	\end{pmatrix}}_{V^T}
\end{align}

\subsubsection{Appplication, inference of signals}
It is said that SVD can pick out "interesting" signals from data. I'll illustrate this with a simple toy model. You have a rank-1 correction to a matrix full of noise, you want to infer this correction \& it's strength. This is called the Spiked Wigner model.
\begin{align}
	A = \lambda xx^T + W
\end{align}
where $W \sim \text{GOE}(n)$, this means that on-diagonal entires $W_{ii} \sim \mathcal N(0, 2)$, and off-diaongal entries $W_{ij} = W_{ji} \sim \mathcal N(0,1)$. For such a matrix, the expectation value element-wise yields: $\mathbb E[W] = 0$ and $\mathbb E[W^T W] = \sigma^2_n \mathbb I$.
\begin{align}
	\mathbb E[A^T A] & = \lambda^2 x x^T  x x^T  + \sigma^2 \mathbb I\\
	& = \lambda^2 x^2 \, xx^T + \sigma^2 \mathbb I
\end{align}
The eigensystem for this is
\begin{itemize}
	\item One eigenvector $x$, with eigenvalue $ \lambda^2 \|x|^4 + \sigma^2$
	\item $d-1$ eigenvectors $v$ s.t. $v \perp x$, with eigenvalue $\sigma^2$.
\end{itemize}
Recall the SVD $A = U \Sigma V^T$. the $V$ were the eigenvectors of $A^T A$, which we found contains the signal we wanted to infer. The $\Sigma$ contains the eigenvalues of $A^T A$, which contain information on the strength of the signal $\lambda$ and noise $\sigma^2$.

For those who know random matrix theory, you are probably skeptical due to BBP transition. This calculation doesn't ask whether you can infer $xx^T$ from a single observation of $Y$, it's given infinite observations here's what you expect

\subsection{Matrix Calculus}
In the next week you'll have to optimize your neural network. Optimization scheme rely on gradient access to your target function, so you'll have to learn how to compute derivatives of vectors \& such.
\begin{sidework}
	A tiny note on notation. Consider a vector $\mathbf x : \mathbb R \to \mathbb R^{d \times 1}$ (for example a trajectory), where $\mathbf x = (x_1, ..., x_d)^T$ : 
	\begin{align}
		\frac{d \mathbf x}{d t} \equiv \begin{pmatrix}
			\frac{\partial x_1}{\partial t} \\ \vdots \\ \frac{\partial x_d}{\partial t}
		\end{pmatrix}.
	\end{align}
	Now consider the scalar field $\phi : \mathbb R^d \to \mathbb R$
	\begin{align}
		\frac{\partial \phi}{\partial \mathbf x} \equiv \begin{pmatrix}
			 \frac{\partial \phi}{\partial x_1} &   ...  & \frac{\partial \phi}{\partial x_d}  
		\end{pmatrix} = \left( \nabla \phi \right)^T \in \mathbb R^{1 \times d}.
	\end{align}
	A nice thing about the notation is for $\phi = \phi(\mathbf x(t))$, computing $\frac{\partial \phi}{\partial t} = \langle \nabla \phi , \frac{d \mathbf x}{d t} \rangle = \frac{\partial \phi}{\partial \mathbf x} \frac{d \mathbf x}{d t}$ becomes very natural. Another bonus is it mirrors the physics notation, the derivative of a contravariant vector $\frac{\partial}{\partial x^\alpha}$ transforms as a covariant vector $\partial_\alpha$. I'll note most statisticians don't use this notation, they treat $\partial \phi / \partial \mathbf x$ as a column vector...
	
	Now consider the vector field $\mathbf y : \mathbb R^d \to \mathbb R^n$ (for example change of coordinates)
	\begin{align}
		\frac{\partial \mathbf y}{\partial \mathbf x} \equiv 
		\begin{pmatrix}
			- & \frac{\partial  y_1}{\partial \mathbf x}  & -\\
			  & \vdots  & \\
			  - & \frac{\partial  y_n}{\partial \mathbf x} & -
		\end{pmatrix} \in \mathbb R^{n \times d}
	\end{align}
	this is also known as the Jacobian. The notation is written s.t. $\frac{\partial \mathbf y}{\partial \mathbf x} \mathbf x$ is a sensible matrix multiplication. 
	\\
	Finally consider a matrix $\mathbf M \in \mathbb R^{m \times n}$
	\begin{align}
		\frac{\partial \mathbf M}{\partial t} & \equiv \begin{pmatrix}
			\frac{\partial M_{11}}{\partial t} & \hdots & \frac{\partial M_{1n}}{\partial t}\\
			\vdots & \ddots & \vdots \\
			\frac{\partial M_{m 1}}{\partial t} & \hdots  & \frac{\partial M_{mn}}{\partial t}
		\end{pmatrix}\\
		\frac{\partial t}{\partial \mathbf M} & = \begin{pmatrix}
			\frac{\partial t}{\partial M_{11}} & \hdots & \frac{\partial t}{\partial M_{1n}}\\
			\vdots & \ddots & \vdots \\
			\frac{\partial t}{\partial M_{m 1}} & \hdots  & \frac{\partial t}{\partial M_{mn}}
		\end{pmatrix}
	\end{align}
	Derivatives of matrices against vectors (and vice versa) (and higher order tensors), are defined in terms of index notation.
\end{sidework}

Here are some simple facts. 

\subsubsection{Derivatives of scalar forms}
Let $\mathbf a, \mathbf b$ be constants
\begin{align}
	\frac{\partial (\mathbf a^T \mathbf x)}{\partial \mathbf x} & = \frac{\partial (\mathbf x^T \mathbf a)}{\partial \mathbf x} = \mathbf a^T & \mathbf a \text{ constant}\\
	\frac{\partial(\mathbf x^T \mathbf M \mathbf x)}{\partial \mathbf x}  & = \mathbf x^T (\mathbf M + \mathbf M^T) \\
	\frac{\partial(\mathbf a^T \mathbf M \mathbf b)}{\partial \mathbf M} & = \mathbf a  \mathbf b^T\\
	\frac{\partial (\mathbf a^T \mathbf M^T \mathbf b)}{\partial \mathbf M} & = \mathbf b \mathbf a
\end{align}
Note due to my notation, the $\frac{\partial }{\partial \mathbf x}$ terms have a relative transpose compared to Sam Roweis' notes.

\subsubsection{Derivatives of vector forms}
\begin{align}
	\frac{\partial \mathbf x}{\partial \mathbf x} & = \mathbb I\\
	\frac{\partial(\mathbf M \mathbf x)}{\partial \mathbf x} & = \mathbf M
\end{align}





\begin{problem}
	Derive the back-propagation for a two layer neural network. That is given
	\begin{align}
		\mathcal L & = (y - \hat y)^2\\
		\hat y & = \frac{1}{N}\sum_{i=1}^N a_i \, \sigma( \mathbf{w}_i^T  \mathbf x + b_i)
	\end{align}
	where $y, \hat y, a_i, b_i \in \mathbb R$ are scalars, and $\mathbf w_i, \mathbf x \in \mathbb R^d$ are vectors. Note $\mathbf w_i$ is not the entry of a vector, there are $i = 1,..., N$ $\mathbf w_i$ vectors.
	\\
	\\
	Compute 
	\begin{align}
		\frac{\partial \mathcal L}{\partial \mathbf w_i}
	\end{align}
	this is the notation for the gradient w.r.t. $\mathbf w_i$.
\end{problem}


\subsubsection{Matrix Inversions}
\begin{fact}[Sherman-Morrison] Let $A$ be an invertible square matrix, and $u,v$ be vectors.  
	\begin{align}
		(A + uv^T)^{-1} = A^{-1} + \frac{A^{-1} u v^T A^{-1}}{1 + v^T A^{-1} u} 
\end{align}
\end{fact}

\noindent
\begin{sidework}
	\emph{Proof:} Here's a constructive proof. Say you want to solve for $x$
\begin{align}
	(A + uv^T) x & =  y\\
	x & = A^{-1} y - A^{-1} u v^T x
\end{align}
Notice 
\begin{align}
	v^T x & = v^T A^{-1} y - v^TA^{-1} uv^T x\\
	(1 + v^T A^{-1} u)v^T x & = v^T A^{-1}y\\
	v^T x & = \frac{v^T A^{-1}}{1 + v^T A^{-1} u} y 
\end{align}
Therefore
\begin{align}
	x = \left(A^{-1} - \frac{A^{-1} u v^T A^{-1}}{1 + v^T A^{-1} u} \right) y
\end{align} 
\qed
\end{sidework}

The idea is that a rank one perturbation $uv^T$ to a full rank matrix $A$, yields an inverse which is a rank one perturbation to $A^{-1}$.

\begin{fact}
	[Woodbury] A generalization of the previous fact is
	\begin{align}
		(A + UCV)^{-1} = A^{-1} + A^{-1} U (C^{-1} + V A^{-1} U)^{-1} V A^{-1}
	\end{align}
	where $A$ is $n \times n$, $C$ is $k \times k$, $U$ is $n \times k$ and $V$ is $k \times n$.
\end{fact}
Proof left as exercise.



\subsection{Time Complexity}
Since we're talking about these things in the context of a computational class, it'll be good to recap the time complexity of such algorithms. Just keep these in the back of your mind. 


\begin{itemize}
	\item Matrix multiplication: $\mathcal O(n^{2.8})$.
	\item Matrix inverse implemented in \texttt{numpy.linalg.solve}: $\mathcal O(n^3)$.
	\item SVD for a $n \times m$ matrix (s.t. $n \leq m$) : $\mathcal O(m n^2)$.
	\item Determinant $\mathcal O(n^3)$ 
\end{itemize}
As a final note, the time complexity of an algorithm doesn't translate to the actual run time of an algorithm.
\\
\\
See \url{https://en.wikipedia.org/wiki/Computational_complexity_of_mathematical_operations#Matrix_algebra} for more information.






\begin{thebibliography}{9}
\bibitem{svd}
Michael Hutchings, Notes on singular value decomposition for Math 54, \url{https://math.berkeley.edu/~hutching/teach/54-2017/svd-notes.pdf}.

\bibitem{gundersen}
Gregory Gundersen, Singular Value Decomposition as Simply as Possible, \url{https://gregorygundersen.com/blog/2018/12/10/svd/}

\bibitem{lamport94}
Leslie Lamport (1994) \emph{\LaTeX: a document preparation system}, Addison
Wesley, Massachusetts, 2nd ed.
\end{thebibliography}







\newpage

\section{Review of Probability}

\begin{definition}
	[Conditional Probability]
	\begin{align}
		\mathbb P[ A | B] = \frac{\mathbb P[A \cap B]}{P[B]}
	\end{align}
\end{definition}
Notice that $\mathbb P[A \cap B] = \mathbb P[B \cap A]$, this allows us to related $\mathbb P[A|B]$ and $\mathbb P[B | A]$.
\begin{align}
	\mathbb P[ A | B] & = \frac{\mathbb P[A \cap B]}{P[B]}\\
	& = \frac{\mathbb P[B \cap A]}{P[B]}\\
	\Aboxed{\mathbb P[ A | B] & = \frac{\mathbb P[B | A] \, \mathbb P[A]}{\mathbb P[B]} }
\end{align}
This is \textbf{Bayes' Formula}.



\begin{definition}
	[Probability Density Function] A function with the following properties is a \textbf{probability density}
	\begin{itemize}
		\item  Positive: $p : \mathcal X \to \mathbb R_{\geq 0 }$
		\item  Normalized: $\int_{\mathcal X} p(x) \, dx = 1$
	\end{itemize}
	It is interpreted as the probability of observing an event $A \subset \mathcal X$  as
	\begin{align}
		\mathbb P[x \in A] = \int_{A \subset \mathcal X} p(x) \, dx
	\end{align}
\end{definition}

The nice part of densities is that you can compute statistics with that. I.e. what's the mean, variance.
\begin{align}
	\mathbb E_{x \sim p} [f(x)]  =  \int_{\mathcal X} f(x) \, p(x) \, dx
\end{align}

\begin{definition}
	[Characteristic Function] Consider the probability distribution $p_X$. It has an associated \textbf{characteristic function} $\varphi_X$  which is it's Fourier Transform
	\begin{align}
		\varphi_X(k) = \int_{\mathbb R} e^{ikx} p(x) \, dx = \mathbb E_{x \sim p}[e^{ikx}]
	\end{align}
\end{definition}
\input{statistics.tex}


\newpage
\part{Supervised Learning}
\section{PyTorch 101}
I have a PyTorch tutorial in 


\section{Tricks for Training Neural Networks}

If you've been playing around with the homework, you might find that just running a vanilla MLP doesn't get good performance. Figuring out why certain tricks work (vs don't work) is a bit like research, so be patient with yourself.

\subsubsection{Whitening}
The following analysis will assume the data distribution $\mathcal D = \{x^{(i)}\}_i \sim^{iid} p_{data}$, has vanishing $\mathbb E[x^{(i)}] = 0$ and covariance $\mathbb E[x^{(i)}x^{(i)T} ] = \sigma^2 \mathbb I$. You might ask, is it this a fair assumption, and the answer is totally. You can always transform data to have these properties.

Consider the data matrix $X \in \mathbb R^{b \times d}$ (each row is a $d$-dimensional data point)
\begin{align}
	X \equiv \begin{pmatrix}
		- & x^{(1)} & -\\
		  & \vdots & \\
		- & x^{(b)} & -
	\end{pmatrix}
\end{align}



\subsection{Training Dynamics}
Thinking of back-propagation as a high-level chain rule is not enough to get a neural network to train. For example, here's some food for thought...
\begin{itemize}
	\item We want to $\arg\min_\theta \mathcal L(\theta)$, what scheme do you implement? Note that $\mathcal L(\theta)$ implicitly depends on the data and model architecture, so perhaps that'll affect your answer.
	\item In the model architecture, how do you initialize the model weights?
\end{itemize}

\subsection{Whitening}


\subsection{Back propagation}
Using the rules from our matrix cheat sheet. Let $x \in \mathbb R^{m}$. Consider a linear layer 
\begin{align}
	\text{Linear}& : \mathbb R^{m} \to \mathbb R^{n}\\
	z(x) = \text{Linear}(x) & = Wx + b\\
	\frac{\partial}{\partial W_{ij}}  \text{Linear}(x)& = x_j\\
	\frac{\partial}{\partial W_{ij}}  \sigma(z)
\end{align}



\subsection{Weight Initialization}
As you propogat






\begin{thebibliography}{9}
\bibitem{recipe}
Andrej Karpathy, A Recipe for Training Neural Networks, \url{https://karpathy.github.io/2019/04/25/recipe/}.


\end{thebibliography}









\newpage

\section{Convolutional Neural Networks}

\section{Graph Neural Networks}
\begin{definition}
	[Graph] A graph $G = (V,E)$ is two sets. 
	\begin{itemize}
		\item $V = \{v_1, ..., v_N\}$ are the vertexes / nodes. At each node $v_i$ a vector of features is attached $\mathbf h_i \in \mathbb R^d$. 
		\item $E \subseteq V \times V$ are the dedges.
	\end{itemize}
	We represent connectivity using an adjacency matrix
	\begin{align}
		A_{ij} = \begin{cases}
			1 & (v_i, v_j) \in E\\
			0 & \text{else}
		\end{cases}
	\end{align}
\end{definition}






\section{Training Large Models}
\subsection{Transformers}
\subsection{$\mu$P Optimizer}
\section{Geometric Deep Learning}


\newpage
\part{Probabilistic Inference}
\section{Variational Inference}
There were a couple of bottlenecks on MCMC
\begin{enumerate}
	\item If the query time for the likelihood is quite long, then sampling the distribution will take very long.
	\item In multimodal distributions, we have no guarantees when the chains will mix / find other mode, so MCMC may fail in that regime. Note, multimodal is NP-hard so this new technique won't entirely solve it.
\end{enumerate}
Instead, what if you attempted to approximate the target distribution, in such a way it was easy to sample? As physicists, this must sound familiar to you. Recall the variational principle from quantum mechanics.
\begin{align}
	\langle \psi_\theta | H | \psi_\theta \rangle \geq E_{gs}
\end{align}
where $\psi_\theta$ is a trial wave function. You $\arg \min_\theta \langle \psi_\theta | H | \psi_\theta \rangle$ in an attempt to find the ground state. But let me restate this in a more statistical language. You parameterize your probability distribution (wave function) via a neural network, and you adjust the parameters to minimize the discrepancy between  the parameterized distribuiton \& the target distribution.

This now raises questions
\begin{enumerate}
	\item Are there "distance" measurements between probability distributions? Which one should I pick as a loss function?
	\item How can parameterize a probability distribution in a very flexible way s.t. (1) I can sample it easily, (2) perhaps I can even evaluate it's log-probability?
\end{enumerate}



\subsection{Distance of Probability Measures}
There are couple ways to measure the distance between probability measures. Each is good for a certain context. A (very) non-exhaustive list
\begin{itemize}
	\item Total Variation distance 
		\begin{align}
			\| p - q\|_{TV} = \frac{1}{2} \int |p(x) - q(x) | \, dx
		\end{align}
		\emph{Comments:} It's an $L_1$ bound on the distributions, so if you don't care about the moments but making sure the spaces are close, then it's good. However, you can NOT bound moments using this...
	\item Kullback-Leibler divergence
		\begin{align}
			\text{KL}(p\|q) = \int p(x) \, \log \frac{p(x)}{q(x)} \, dx
		\end{align}
		Note this is not symmetric, hence why we don't call it a distance but instead a divergence.
		
		\emph{Comments:} It's easy to numerically compute. You can rewrite it as
		\begin{align}
			\text{KL}(p \| q) = \mathbb E_{x\sim p} [\log p(x)] - \mathbb E_{x\sim p}[\log q(x)].
		\end{align}
		In computation, you usually have access to log-probabilities, and you can approximate the expectation using a sampling technique. Also, the KL is related to the TV distance by Pinsker's inequality: $\|p - q\|_{TV} \leq \sqrt{\frac{1}{2} \text{KL}(p \| q)}$. 
	\item Wasserstein Distance
		\begin{align}
			W_p(p,q) = \sup_{\gamma \in \Gamma} \int \| x- y \|^p \, d\gamma(x,y)^{1/p}
		\end{align}
		where $\Gamma$ is the set of couplings on $p,q$. A coupling $\gamma(p,q)$ is defined a joint probability distribution s.t. it marginalizes to recover $p,q$; $\int \gamma(p(x), q(y)) dx = q(y)$ and vice versa.
		
		\emph{Comments:} This is considered the most natural way to compare distributions. You can also bound moments, unlike the TV distance.
\end{itemize}

Any of these measurements will equal zero i.f.f. the two distributions are equal, and  they're all non-negative. Making them good loss functions for neural networks. As hinted earlier, the KL Divergence will be preferred for computational easyness. The question becomes which argument do the, parameterized distribution $p_\theta$ and target distribution $\pi$, go?
\begin{align}
	\KL(p_\theta \| \pi) & = \int p_\theta(x) \log \frac{p_\theta(x)}{\pi(x)} \, dx =  \mathbb E_{x \sim p_\theta} [\log p_\theta(x) ] - \mathbb E_{x \sim p_\theta} [\log \pi(x)]\\
	\KL(\pi \| p_\theta) & = \mathbb E_{x \sim \pi} [\log \pi(x)] - \mathbb E_{x \sim \pi}[\log p_\theta(x)]
\end{align}



\section{Normalizing Flows}
To get variational inference to work better than just MCMC, here's a couple things I want
\begin{enumerate}
	\item I want IID samples
	\item Ability to evaluate the normalized log-probability of the target distribution.
\end{enumerate}
Points (2) and (3) give us a hint as to a potential solution. Imagine constructing a map between an easy to evaluate distribution (i.e. Gaussian) and your target distribution, then transporting sampling according to this map. A potential problem is an approximate map will yield bias in your final answer (I'll touch upon this later).

An example of this is the Box-Muller transformation. Your base distribution $\nu = \text{Unif}(0,1]^2$, and you construct a map which maps you to a target distribution $\pi = \mathcal N(0, \mathbb I_2)$. So if you sample $u \sim \nu$, we can use an invertible function $f(u) = z \sim \pi$. Since $f(u)$ actually is equal in distribution to $\pi$ then if you can sample $\nu$ iid (which you can), then you automatically sample $\pi$ iid.  For completeness, here's the map
\begin{align}
	\begin{pmatrix}
		z_1 \\ z_2
	\end{pmatrix} = f \begin{pmatrix}
		u_1 \\ u_2
 	\end{pmatrix} = \begin{pmatrix}
 		\sqrt{-2 \log u_1} \cos(2\pi u_2) \\ \sqrt{-2 \log u_1} \sin(2\pi u_2)
 	\end{pmatrix} 
 \end{align}
Evaluating the log-prob of the samples is trivial for a gaussian (just plug the realization $z$ of the target distribution into $\propto e^{-x^2/2}$). But what if you didn't have oracle access to the target distribution? Consider how you'd evaluate probability distributions under a change of variables
\begin{align}
	\int \pi(z) dz & = \int \nu(u) du = 1 \\
	\int  \pi(f(u)) \left | \det \frac{d f}{du} \right | \, du & = \int \nu(u) du\\
	\pi(f(u)) \left | \det \frac{d f}{du} \right | & = \nu(u)\\
	\log \pi(z) & =  \log \nu(f^{-1}(z))  - \log | \det \frac{\partial f}{\partial u }|_{u = f^{-1}(z)} |
\end{align}
This means you can evaluate $\log \pi(z)$ (normalization included) by moving the generated samples back to the base distribution.

The idea of normalizing flows is to parameterize the change-of-variables / push-forward via a neural network. We usually use the base distribution $z \sim \nu = \mathcal N(0, \mathbb I_d)$ and the target distribution $x \sim \pi$. 
\begin{align}
	x & = f_L \circ f_{L-1} \circ ... \circ  f_1(z)\\
	\log  \pi(x) & = \log \pi_0(z_0) - \sum_{i=1}^L \log \left | \det \frac{df_i}{dz_{i-1}}\right|
\end{align}
When we parameterize $f$ with a neural network, we need to construct layers that are:
\begin{enumerate}
	\item Easily invertible. Keep in mind that matrix inverses are more expensive than matrix multiplications.
	\item The Jacobian is easily computable
\end{enumerate}



\subsection{Architectures}
\subsubsection{RealNVP}
Real Non-Volume Preserving implements the following function $\mathbf x \mapsto \mathbf y$. It splits into two section, $\mathbf x_{1:d} \equiv (x_1, ..., x_d)$ stays in the same, and $\mathbf x_{d+1: D}$ is modified
\begin{align}
	\mathbf y = \begin{pmatrix}
		\mathbf y_{1:d}\\ \mathbf y_{d+1: D}
	\end{pmatrix}  = \begin{pmatrix}
		\mathbf x_{1:d}\\ \mathbf x_{d+1:D} \odot  \exp(s(\mathbf x_{1 :d})) + t(\mathbf x_{1:d})
	\end{pmatrix}
\end{align}
where $\odot$ is element wise multiplication, and the $s, t$ functions are applied element-wise as well. To solve for the inverse treat everything as scalars, and you get
\begin{align}
	\mathbf x = \begin{pmatrix}
		\mathbf x_{1:d} \\ \mathbf x_{d+1:D}
	\end{pmatrix} = \begin{pmatrix}
		\mathbf x_{y:d} \\ (\mathbf y_{d+1:D} - t(\mathbf y_{1:d})) \odot \exp(- s(\mathbf y_{1:d}))
	\end{pmatrix}
\end{align}
What's really nice is that you don't need to invert $s,t$, so they themselves can be anything-- we will parameterize them by neural networks. As for the Jacobian...
\begin{align}
	\frac{d\mathbf y}{d \mathbf x} & = 
	\begin{pmatrix}
		\frac{d\mathbf y_{1:d}}{d \mathbf x_{1:d}} & \frac{d \mathbf y_{1:d}}{d \mathbf x_{d+1:D}}\\
		\frac{d\mathbf y_{d+1:D}}{d \mathbf x_{1:d}} & \frac{d\mathbf y_{d+1:D}}{d \mathbf x_{d+1:D}}
	\end{pmatrix}\\
	& = \begin{pmatrix}
		\mathbb I_d  & 0 \\
		\frac{d\mathbf y_{d+1:D}}{d \mathbf x_{1:d}}  & D 
	\end{pmatrix}
\end{align}
where $[D]_{ii} = \exp([s(\mathbf x_{1:d})]_i)$.
You can see that this split $\mathbf y = (\mathbf y_{1:d} , \mathbf y_{d+1: D})$ was to make the Jacobian triangular, allowing for a speedy evaluation. In particular just the determinant matters, so the nasty bottom left entry disappears. Since the resulting Jacobian is diagonal, 
\begin{align}
	\log \det \frac{d\mathbf y}{d\mathbf x} = \log \prod_{i=1}^d \exp([s(\mathbf x_{1:d}]_i) = \sum_{i=1}^d [s(\mathbf x_{1:d})]_i
\end{align}





















\section{Review of Stochastic Differential Equations}
You've probably heard of SDEs before, but they aren't covered in the main-stream physics education. So I'll attempt to do a brief introduction. 

There was a botanist studying pollen grains in water. He noticed the motion was jittery, moving randomly in all directions. You can imagine a heuristic model being
\begin{align}
	X_{t+h} = X_t + h^\alpha \, z_t
\end{align}
where $z_t \sim \mathcal N(0, \mathbb I)$ (iid at every time $t$) is random noise, and $h$ is the step size (according to the time-discretization) to the power $\alpha$. In an attempt to find a continuous time model in the limit $h \to 0$ (discretization goes to zero), I'll recurse to time zero.
\begin{align}
	X_{t} &= X_{t-h} + h^\alpha \, z_{t-h}\\
	& = X_{t-2h} + h^\alpha \, (z_{t-2h} + z_{t-h})\\
	& = X_{t-3h} + h^\alpha \, (z_{t-3h} + z_{t-2h} + z_{t-h})\\
	& = X_0 + h^\alpha \sum_{n=1}^{t/h } z_{t - nh}
\end{align}
Since we're physicists, let's center the initial position $X_0 = 0$. We now note that
\begin{align}
	h^\alpha \, \sum_{n=1}^{t/h  + 1} z_{t - nh} \sim   \mathcal N(\left(0, h^{2\alpha - 1} t \right)
\end{align}
To keep the model independent on the size of the discretization, I'll choose $\alpha = 1/2$. Leaving us with
\begin{align}
	X_t - X_0 \sim \mathcal N(0, t)
\end{align}
This was quite heuristic, but we have some take aways. When making an infinitesimal that behaves randomly, it has units $\sqrt{dt}$.

Now that we have some intuition for the system, we can develop something more rigorous.
\begin{definition}
	[Weiner Process / Brownian Motion] Brownian motion $(W_t)_{t\geq 0}$ is a stochastic process such that
	\begin{enumerate}
		\item Initializes at zero: $W_0 = 0$
		\item Normal increments: $W_t - W_s \sim \mathcal N(0,(t-s) \mathbb I)$, for $ 0 \leq s \leq t$.
		\item Independent increments: $W_{t_1} - W_{t_0}$ is independent from $W_{t_i} - W_{t_j}$.
	\end{enumerate}
\end{definition}
The idea of a stochastic differential equations is to extend the dynamics of ODEs to the dynamics where you have random fluctuations of force. Such things are no-where differentiable, so how can we recover a derivative-esq operation w/o using a derivative? Well ODEs  have that
\begin{align}
	\frac{dX_t}{dt} = \mu_t(X_t) \implies X_{t+h} = X_t + h\, u_t(X_t) + \mathcal O(h^2)
\end{align}
Similarly for an SDE (ODE with stochastic fluctuations)
\begin{align}
	X_{t+h} = X_t + X_t + h u_t(X_t)  + (W_{t+h} - W_t) \, \sigma_t(X_t) + \mathcal O(h^{3/2}) \label{eqn:SDETrue}
\end{align}
The $\mathcal O(h^{3/2})$ is due to fluctuations on the order $h \, (W_{t+h} - W_t)$, as we've noted that $W_{t+h} - W_t$ is order $\sqrt{h}$. 

For brevity, we'll use a shorthand for \ref{eqn:SDETrue}
\begin{align}
	dX_t = \mu_t(X_t) \, dt + \sigma_t(X_t)\, dW_t
\end{align}

\begin{theorem}
	[Fokker-Planck Equation]
	Consider the stochastic differential equation
	\begin{align}
		dX_t & = \mu_t(X_t) \, dt + \sigma_t dW_t\\
		X_0 & \sim p_0 & \text{Boundary condition}
	\end{align}
	where $\mu_t : [0,1] \times \mathbb R^d \to \mathbb R^d$ and $\sigma_t : [0,1] \to \mathbb R^d$ are deterministic functions. Then the corresponding probability distribution $X_t \sim p_t$ solves a partial differential equation of the following form
	\begin{align}
		\partial_t p_t(x) & = - \nabla \cdot (\mu_t\, p_t) + \frac{\sigma_t^2}{2} \Delta p_t\\
		p_{t=0} & = p_0 & \text{Boundary condition}
	\end{align}
\end{theorem}
\begin{sidework}
	\emph{Proof:}  Since $X_t$ is a random variable, it has a corresponding probability density function. I'll notate this as $p_t$. Now you need to show that $p_t$ have a the corresponding time evolution. The trick to do this, is to recall the trick you employ when you show something is secretly a delta function. You would integrate it against a test function $f(x)$ and show it behaved as expected. We'll do the same thing.
\begin{align}
	\partial_t \mathbb E[f(X_t)] & = \lim_{h \to 0} \frac{1}{h} \mathbb E[f(X_{t+h}) - f(X_t)] \\
	& = \lim_{h \to 0} \mathbb E[\nabla f^T\, u_t(X_t) + \frac{\sigma^2_t}{2} \Delta f(X_t) + \mathcal O(h)]\\
	& = \int \nabla f^T (x) u_t(x) p_t(x) +  \frac{\sigma^2_t}{2} \Delta f(x)  p_t(x) \, dx\\
	& = \int -f(x) \, \nabla \cdot (u_t(x) p_t(x)) + f(x) \frac{\sigma^2_t}{2} \Delta p_t(x)\, dx
\end{align}
On the LHS
\begin{align}
	\partial_t \mathbb E[f(X_t)] = \int f(x) \partial_t p_t(x) \, dx
\end{align}
Put LHS = RHS, and you're done. $\hfill \Box$
\end{sidework}
As a side note, this proof is typically done using Ito's lemma, and (I think) it holds in weak-convergence. This proof uses the Euler Maruyama, and taking $h \to 0$ gives strong convergence (which in turn implies weak-convergence). 







\section{Score Based Diffusion}

\section{Stochastic Interpolants}





\end{document}