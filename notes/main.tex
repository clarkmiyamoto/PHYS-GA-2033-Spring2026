%%% Document Formatting
\documentclass[12pt,fleqn]{article}
\usepackage[a4paper,
            bindingoffset=0.2in,
            left=0.75in,
            right=0.75in,
            top=0.8in,
            bottom=0.8in,
            footskip=.25in]{geometry}
\setlength\parindent{10pt} % No indent

%%% Imports
% Mathematics
\usepackage{amsmath} % Math formatting
\numberwithin{equation}{section} % Number equation per section
\DeclareMathOperator{\Tr}{Tr}


\newtheorem{theorem}{Theorem}
\newtheorem{definition}{Definition}
\newtheorem{proof}{Proof}
\newtheorem{lemma}{Lemma}
\newtheorem{example}{Example}
\newtheorem{fact}{Fact}


\usepackage{amsmath}
\usepackage{amsfonts} % Math fonts
\usepackage{amssymb} % Math symbols
\usepackage{mathtools} % Math etc.
\usepackage{slashed} % Dirac slash notation
\usepackage{cancel} % Cancels to zero
\usepackage{empheq}
\usepackage{breqn}

\newcommand{\expect}[1]{\mathbb{E}\left[#1\right]}
\newcommand{\qed}{\hfill$\square$}


% Visualization
\usepackage{graphicx} % for including images
\graphicspath{ {} } % Path to graphics folder
\usepackage{tikz}



%%% Formating
\usepackage{hyperref} % Hyperlinks
\hypersetup{
    colorlinks=true,
    linkcolor=blue,
    filecolor=magenta,      
    urlcolor=cyan,
    pdftitle={Overleaf Example},
    pdfpagemode=FullScreen,
    }
\urlstyle{same}

\usepackage{mdframed} % Framed Enviroments
\newmdenv[ 
  topline=false,
  bottomline=false,
  skipabove=\topsep,
  skipbelow=\topsep
]{sidework} %% Side-work

\usepackage{lipsum} % Lorem Ipsum example text

%%%%% ------------------ %%%%%
%%% Title
\title{Machine Learning for Physicists: Recitation Notes}
\author{Clark Miyamoto (cm6627@nyu.edu)}
\date{\today}
\begin{document}

\maketitle

\tableofcontents


\newpage


\input{linearalgebra.tex}
\newpage

\section{Review of Probability}

\begin{definition}
	[Conditional Probability]
	\begin{align}
		\mathbb P[ A | B] = \frac{\mathbb P[A \cap B]}{P[B]}
	\end{align}
\end{definition}
Notice that $\mathbb P[A \cap B] = \mathbb P[B \cap A]$, this allows us to related $\mathbb P[A|B]$ and $\mathbb P[B | A]$.
\begin{align}
	\mathbb P[ A | B] & = \frac{\mathbb P[A \cap B]}{P[B]}\\
	& = \frac{\mathbb P[B \cap A]}{P[B]}\\
	\Aboxed{\mathbb P[ A | B] & = \frac{\mathbb P[B | A] \, \mathbb P[A]}{\mathbb P[B]} }
\end{align}
This is \textbf{Bayes' Formula}.



\begin{definition}
	[Probability Density Function] A function with the following properties is a \textbf{probability density}
	\begin{itemize}
		\item  Positive: $p : \mathcal X \to \mathbb R_{\geq 0 }$
		\item  Normalized: $\int_{\mathcal X} p(x) \, dx = 1$
	\end{itemize}
	It is interpreted as the probability of observing an event $A \subset \mathcal X$  as
	\begin{align}
		\mathbb P[x \in A] = \int_{A \subset \mathcal X} p(x) \, dx
	\end{align}
\end{definition}

The nice part of densities is that you can compute statistics with that. I.e. what's the mean, variance.
\begin{align}
	\mathbb E_{x \sim p} [f(x)]  =  \int_{\mathcal X} f(x) \, p(x) \, dx
\end{align}

\begin{definition}
	[Characteristic Function] Consider the probability distribution $p_X$. It has an associated \textbf{characteristic function} $\varphi_X$  which is it's Fourier Transform
	\begin{align}
		\varphi_X(k) = \int_{\mathbb R} e^{ikx} p(x) \, dx = \mathbb E_{x \sim p}[e^{ikx}]
	\end{align}
\end{definition}
\input{statistics.tex}




\section{Double Descent}


\subsection{Soft Inductive Biases}
Another way to conceptualize this is \textbf{soft inductive biases} (see Andrew Gordon Willson's paper \url{https://arxiv.org/pdf/2503.02113}). 




\section{Training Large Models}
\subsection{Transformers}
\subsection{$\mu$P Optimizer}
\section{Geometric Deep Learning}

\section{Review of Stochastic Differential Equations}
You've probably heard of SDEs before, but they aren't covered in the main-stream physics education. So I'll attempt to do a brief introduction. 

There was a botanist studying pollen grains in water. He noticed the motion was jittery, moving randomly in all directions. You can imagine a heuristic model being
\begin{align}
	X_{t+h} = X_t + h^\alpha \, z_t
\end{align}
where $z_t \sim \mathcal N(0, \mathbb I)$ (iid at every time $t$) is random noise, and $h$ is the step size (according to the time-discretization) to the power $\alpha$. In an attempt to find a continuous time model in the limit $h \to 0$ (discretization goes to zero), I'll recurse to time zero.
\begin{align}
	X_{t} &= X_{t-h} + h^\alpha \, z_{t-h}\\
	& = X_{t-2h} + h^\alpha \, (z_{t-2h} + z_{t-h})\\
	& = X_{t-3h} + h^\alpha \, (z_{t-3h} + z_{t-2h} + z_{t-h})\\
	& = X_0 + h^\alpha \sum_{n=1}^{t/h } z_{t - nh}
\end{align}
Since we're physicists, let's center the initial position $X_0 = 0$. We now note that
\begin{align}
	h^\alpha \, \sum_{n=1}^{t/h  + 1} z_{t - nh} \sim   \mathcal N(\left(0, h^{2\alpha - 1} t \right)
\end{align}
To keep the model independent on the size of the discretization, I'll choose $\alpha = 1/2$. Leaving us with
\begin{align}
	X_t - X_0 \sim \mathcal N(0, t)
\end{align}
This was quite heuristic, but we have some take aways. When making an infinitesimal that behaves randomly, it has units $\sqrt{dt}$.

Now that we have some intuition for the system, we can develop something more rigorous.
\begin{definition}
	[Weiner Process / Brownian Motion] Brownian motion $(W_t)_{t\geq 0}$ is a stochastic process such that
	\begin{enumerate}
		\item Initializes at zero: $W_0 = 0$
		\item Normal increments: $W_t - W_s \sim \mathcal N(0,(t-s) \mathbb I)$, for $ 0 \leq s \leq t$.
		\item Independent increments: $W_{t_1} - W_{t_0}$ is independent from $W_{t_i} - W_{t_j}$.
	\end{enumerate}
\end{definition}
The idea of a stochastic differential equations is to extend the dynamics of ODEs to the dynamics where you have random fluctuations of force. Such things are no-where differentiable, so how can we recover a derivative-esq operation w/o using a derivative? Well ODEs  have that
\begin{align}
	\frac{dX_t}{dt} = \mu_t(X_t) \implies X_{t+h} = X_t + h\, u_t(X_t) + \mathcal O(h^2)
\end{align}
Similarly for an SDE (ODE with stochastic fluctuations)
\begin{align}
	X_{t+h} = X_t + X_t + h u_t(X_t)  + (W_{t+h} - W_t) \, \sigma_t(X_t) + \mathcal O(h^{3/2}) \label{eqn:SDETrue}
\end{align}
The $\mathcal O(h^{3/2})$ is due to fluctuations on the order $h \, (W_{t+h} - W_t)$, as we've noted that $W_{t+h} - W_t$ is order $\sqrt{h}$. 

For brevity, we'll use a shorthand for \ref{eqn:SDETrue}
\begin{align}
	dX_t = \mu_t(X_t) \, dt + \sigma_t(X_t)\, dW_t
\end{align}

\begin{theorem}
	[Fokker-Planck Equation]
	Consider the stochastic differential equation
	\begin{align}
		dX_t & = \mu_t(X_t) \, dt + \sigma_t dW_t\\
		X_0 & \sim p_0 & \text{Boundary condition}
	\end{align}
	where $\mu_t : [0,1] \times \mathbb R^d \to \mathbb R^d$ and $\sigma_t : [0,1] \to \mathbb R^d$ are deterministic functions. Then the corresponding probability distribution $X_t \sim p_t$ solves a partial differential equation of the following form
	\begin{align}
		\partial_t p_t(x) & = - \nabla \cdot (\mu_t\, p_t) + \frac{\sigma_t^2}{2} \Delta p_t\\
		p_{t=0} & = p_0 & \text{Boundary condition}
	\end{align}
\end{theorem}
\begin{sidework}
	\emph{Proof:}  Since $X_t$ is a random variable, it has a corresponding probability density function. I'll notate this as $p_t$. Now you need to show that $p_t$ have a the corresponding time evolution. The trick to do this, is to recall the trick you employ when you show something is secretly a delta function. You would integrate it against a test function $f(x)$ and show it behaved as expected. We'll do the same thing.
\begin{align}
	\partial_t \mathbb E[f(X_t)] & = \lim_{h \to 0} \frac{1}{h} \mathbb E[f(X_{t+h}) - f(X_t)] \\
	& = \lim_{h \to 0} \mathbb E[\nabla f^T\, u_t(X_t) + \frac{\sigma^2_t}{2} \Delta f(X_t) + \mathcal O(h)]\\
	& = \int \nabla f^T (x) u_t(x) p_t(x) +  \frac{\sigma^2_t}{2} \Delta f(x)  p_t(x) \, dx\\
	& = \int -f(x) \, \nabla \cdot (u_t(x) p_t(x)) + f(x) \frac{\sigma^2_t}{2} \Delta p_t(x)\, dx
\end{align}
On the LHS
\begin{align}
	\partial_t \mathbb E[f(X_t)] = \int f(x) \partial_t p_t(x) \, dx
\end{align}
Put LHS = RHS, and you're done. $\hfill \Box$
\end{sidework}
As a side note, this proof is typically done using Ito's lemma, and (I think) it holds in weak-convergence. This proof uses the Euler Maruyama, and taking $h \to 0$ gives strong convergence (which in turn implies weak-convergence). 







\section{Score Based Diffusion}

\section{Stochastic Interpolants}





\end{document}